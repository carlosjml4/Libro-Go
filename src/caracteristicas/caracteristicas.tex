% Capítulo 2: Características y tipos básicos
\chapter{Características y tipos básicos}

En este capítulo el lector se introducirá de lleno en todos los aspectos básicos
del lenguaje, desde sus principales características a los tipos más básicos,
donde además podrá ver cómo compilar los ficheros fuente utilizando los
compiladores nativos de Go, así como mediante el uso de Makefiles.

\section{Nuestro primer programa: ¡Hola Mundo!}

Sin más preámbulos, y después de contar un poco qué es Go y por qué se ha
realizado este manual, vamos a ver nuestro primer programa en Go: el típico
¡Hola mundo!

\codigo{Hola Mundo}{holamundo}{caracteristicas/holamundo.go}

\section{Garbage Collector\label{Garbage Collector}}

Go posee un Garbage Collector - Recolector de Basura - que identifica cuándo se
deja de utilizar una variable o una instancia concreta, y libera la memoria
asociada de forma automática.\\

Actualmente, el compilador de Go posee un Garbage Collector muy simple pero
efectivo, basado en un ``marcado de barrido'', es decir, marca aquello que puede
ser eliminado, y cuando se activa, se borra.\\

Está en desarrollo un Garbage Collector mucho más avanzado basado en las ideas
del Garbage Collector de IBM\texttrademark
\cite{IBMGC}. Esta nueva
implementación pretende ser muy eficiente, concurrente y de baja latencia, con
lo que nada más detectar que algo sobra, se elimine.

\section{Las bases del lenguaje}

Go está basado en una sintaxis bastante típica en los lenguajes de programación
modernos, con lo que cualquier conocimiento previo de lenguajes como C, serán de
mucha utilidad para seguir el libro.\\

Los ficheros fuente están codificados en UTF-8, lo que implica la posibilidad de
usar cualquier tipo de caracteres tanto latinos, como árabes o chinos en un
mismo fichero fuente, sin ningún tipo de limitación. Los delimitadores de las
declaraciones - o lo que es lo mismo, espacios en blanco no significativos - son
tres: espacios, tabulaciones y saltos de línea.\\

Los identificadores de cualquier variable en un programa escrito en Go, deben de
ser alfanuméricos, incluyendo el caracter `\_', teniendo en cuenta que las
letras y números deben ser aquellas definidas por
Unicode\footnote{\url{http://unicode.org}.}.

\section{Comentarios}

Los comentarios son exactamente los mismos que en C++. Para obtener un
comentario de una línea, utilizamos la combinación de caracteres \emph{//}, mientras
que para un comentario de varias líneas, se usa la combinación \emph{/* */}.

\begin{minipage}{17.1cm}
\begin{lstlisting}[language=go,numbers=none, caption=Comentarios correctos,
label=comentarioscorrectos]
// Esto es un comentario de una linea
/* Esto es un comentario que ocupa mucho
   y se parte en dos lineas. */
\end{lstlisting}
\end{minipage}

Hay que recordar, que los comentarios de varias líneas, no admiten anidamiento,
así pues, el siguiente comentario sería erróneo:

\begin{minipage}{17.1cm}
\begin{lstlisting}[language=go,numbers=none, caption=Comentario incorrecto,
label=comentarioincorrecto]
/* Esto es un comentario que ocupa varias lineas incorrecto.
   /* Con otro comentario anidado de varias lineas, que no es posible hacer.
   	*/
 */
\end{lstlisting}
\end{minipage}

\section{Literales}

Entendemos por \emph{literales} todas aquellas expresiones que representan un
valor concreto, ya sea en forma de cadena de caracteres o en forma numérica.\\

Existen tres tipos de valores literales en Go:

\begin{itemize}
	\item \textbf{Números:} Son aquellos números literales que representan un
	valor concreto sin necesidad de indicar su tamaño.
\begin{minipage}{17.1cm}
\begin{lstlisting}[language=go,numbers=none]
98
0x0FF
2.643e5
\end{lstlisting}
\end{minipage}

	\item \textbf{Strings:} Son aquellas cadenas de caracteres que pueden
	representarse en UTF-8 (o cualquier otra representación Unicode). También
	pueden representarse bytes con \emph{\textbackslash\textbackslash xNN} con
	2 dígitos o con \emph{\textbackslash\textbackslash 012} con 3 dígitos.

\begin{minipage}{17.1cm}
\begin{lstlisting}[language=go,numbers=none]
``Hola mundo\n''
``\\xFF''	// 1 byte
``\\u00FF''	// 1 caracter unicode, 2 bytes en UTF-8
\end{lstlisting}
\end{minipage}

	\item \textbf{Strings puros:} Son cadenas de caracteres que se imprimen tal
		cual son escritas en el código fuente, sin escapar ningún carácter. Se
		representan poniendo la cadena entre dos acentos graves ` \`\  '.

\begin{minipage}{17.1cm}
\begin{lstlisting}[language=go,numbers=none]
`\n\.abc\t\`  ==  ``\\n\\.abc\\t\\''
\end{lstlisting}
\end{minipage}
\end{itemize}

\section{Vistazo rápido de la sintaxis\label{sintaxis}}

La sintaxis de Go es una sintaxis muy familiar si se ha programado antes, aunque
tiene sus propias características y peculiaridades que conviene conocer, pero en
poco tiempo el lector debería reconocer este tipo de sintaxis.\\

A la hora de declarar una variable o un tipo se utiliza la palabra reservada
\emph{var}, seguida del identificador de la variable y terminando por el tipo de
la variable.\\

Veamos esto con un ejemplo, definiendo tres tipos de variables y un tipo \emph{Struct}.

\begin{minipage}{17.1cm}
\begin{lstlisting}[language=go,numbers=none]
var a int         // a es un entero
var b, c *int     // b y c son punteros a enteros
var d []int       // d es un array de enteros

// S es una estructura con dos atributos enteros.
type S struct {
    a, b int
}

var miVar S 	// miVar es una variable de tipo S
\end{lstlisting}
\end{minipage}

Las estructuras de control del programa, también nos resultarán familiares si
hemos trabajado con lenguages imperativos como C o Java, puesto que su
estructura es muy similar. Veamos un ejemplo con un \emph{if} y un \emph{for}.

\begin{minipage}{17.1cm}
\begin{lstlisting}[language=go,numbers=none]
if a == b {
    fmt.Print("a y b son iguales")
} else {
    fmt.Print("a y b son distintos")
}

for i = 0; i < 10; i++ {
	miVar++
}
\end{lstlisting}
\end{minipage}

\nota{En Go los paréntesis en las condiciones del \emph{if} o del
\emph{for} no hay que ponerlas. Sí que son siempre necesarias las llaves, que no
pueden ser omitidas. Además, hay que tener en cuenta que la llave de apertura de
un \emph{if} debe ir en la misma linea que la sentencia, y que el \emph{else},
si lo hubiera, tiene que ir emparejado en la misma linea que el cierre de bloque
del \emph{if}. En la sección \ref{if} en la página \pageref{if} se verá cómo es
la estructura correcta de estas sentencias de control.}

Por último hay que observar en el código anterior la ausencia del carácter `;' como
indicativo del final de sentencia.\\

Cuando Go fue creado, los puntos y coma eran obligatorios en todas las
instrucciones. Finalmente, se acordó que su uso sería opcional, y que sería el
propio compilador el que añadiría los puntos y coma al final de las sentencias
que considerara válidas. Aún así, el uso de puntos y coma en ciertas estructuras
es obligatorio, como puede ser en la separación de las distintas cláusulas de un
bucle \emph{for}.\\

Este esquema sigue las pautas acerca de este tipo de delimitadores del lenguaje
\emph{BCPL}, precursor de \emph{B} y por lo tanto de \emph{C}, y que Go ha decidido
adoptar.\\

Así pues, resumiendo acerca del uso de los puntos y coma:

\begin{itemize}
	\item Son opcionales en todos los programas al final de una sentencia,
	aunque deberían \textbf{no} ponerse. Únicamente son obligatorios a la
	hora de separar los elementos en la cláusula de un bucle \emph{for}
	o elementos en un \emph{if}.
	\item El compilador introducirá automáticamente los puntos y coma, por ello
	no es conveniente ponerlos, al final de una línea no vacía, si lo último que
	se encuentra es:
		\begin{itemize}
			\item Un identificador o un literal.
			\item Alguna de las palabras reservadas: break, continue,
			fallthrough o return.
			\item Alguno de los siguientes tokens: $++$, $--$, ), ], \}.
		\end{itemize}
		\item Se pueden poner los puntos y coma para separar instrucciones en
		una misma línea, pudiendo ser omitido antes de ')' o de '\}'.
\end{itemize}

\section{Tipos}

Go posee únicamente tres tipos básicos: Números, Booleanos y Cadenas de
caracteres.

	\subsection{Números}

	Existen tres tipos numéricos: Enteros, enteros sin signo y números
	flotantes.\\

	Cada uno de estos tipos tiene asociadas una serie de variantes dependiendo
	del número de bits en el que sea almacenado. El cuadro \ref{tiposnumericos}
	muestra un resumen con los tipos numéricos existentes.\\

	\begin{table}[htb]
		\begin{center}
			\begin{tabular}{ccc}
			\textbf{Enteros} & \textbf{Enteros sin signo} & \textbf{Flotantes}\\
			\hline
			int & uint & \\
			int8 & uint8 = byte & \\
			int16 & uint16 & \\
			int32 & uint32 & float32\\
			int64 & uint64 & float64
			\end{tabular}
		\end{center}
	\caption{Tabla de tipos numéricos}
	\label{tiposnumericos}
	\end{table}

	También existe el tipo \emph{uintptr}, que sirve para almacenar número
	enteros lo suficientemente grandes como para necesitar un puntero.\\

	Como puede deducirse de la tabla \ref{tiposnumericos}, el número que acompaña a cada
	nombre de tipo, es el número de bits que ocupa en memoria. Así, podemos
	observar que los números flotantes no tienen representaciones válidas con
	8 y 16 bits.\\

	Los tipos que no tienen asociado ningún número, \emph{int} y \emph{uint},
	se representan con un número de bits igual al ancho de la
	palabra de la máquina en la que se ha compilado el código. De esta forma se
	puede llegar a pensar que en un ordenador de 32-bit, los tipos \emph{int}
	e \emph{int32} son equivalentes, pero \textbf{no} es así. \textbf{Todos
	los tipos de la tabla son distintos}.\\

	Debido a que Go es un lenguaje con tipado fuerte\footnote{\textbf{Tipado
	fuerte}: Un lenguaje tiene tipado fuerte cuando todas las variables tienen
	que declararse con un tipo asociado, y no puede cambiar su tipo en tiempo de
	ejecución.}, no existe conversión implícita de tipos, aunque posteriormente
	veremos cómo pueden realizarse conversiones explícitas.

	\subsection{Bool}

	El tipo \emph{bool} define el tipo booleano usual, con dos constantes
	predefinidas que son: \emph{true} y \emph{false}.\\

	Hay que tener en cuenta, que a diferencia de otros lenguajes, en Go los
	\emph{punteros} y los \emph{enteros} \textbf{no son booleanos}.

	\subsection{String}

	El tipo \emph{string} representa un conjunto invariable de bytes, o lo que es
	lo mismo, texto tal y como lo entendemos. Los strings están delimitados por
	su longitud, no por un carácter nulo como suele ocurrir en la mayoría de
	lenguajes. Esto hace que el tipo \emph{string} sea mucho más seguro y eficiente.\\

	Toda cadena de caracteres representada por el lenguaje, incluidas las
	cadenas de caracteres literales, tienen como tipo \emph{string}.\\

	Como se ha dicho en el primer párrafo, y al igual que ocurre con los números
	enteros, los strings son invariables. Esto significa, que se pueden
	reasignar variables de tipo \emph{string} para que contengan otros
	valores, pero los valores de un literal de este tipo \textbf{no se pueden
	modificar}.\\

	De la misma forma que 5 siempre es 5, "Hola" siempre es "Hola".\\

	En otros lenguajes es posible modificar las cadenas de caracteres, ya sea
	dentro de una variable o de una constante, realizando un manejo explícito de
	la memoria. En Go y gracias a depender de su longitud, un string es
	invariable.\\

	Pese a que todo esto puede parecer incoherente o muy engorroso, las
	librerías de Go poseen un gran soporte para la manipulación de las cadenas
	de caracteres.

\section{Operadores}

	Los operadores en Go, como no podía ser de otra forma, son bastante comunes
	y la mayoría son utilizados en otros lenguajes. Existen operadores binarios,
	que reciben dos operandos, y operadores unarios.

	Tenemos varios tipos de operadores, que se pueden clasificar en 4 categorías
	distintas:

	\begin{itemize}
		\item Operadores aritméticos
		\item Operadores relacionales
		\item Operadores lógicos
		\item Operadores a nivel de bit
	\end{itemize}

	En los siguientes apartados vamos a ver cuáles son los operadores de los que
	disponemos y cuál es su precedencia a la hora de ser evaluados.

	\subsection{Operadores aritméticos}

	Los operadores aritméticos son aquellos que nos permiten realizar
	operaciones aritméticas sobre dos operandos. Son operadores que permiten
	realizar operaciones de suma, resta, multiplicación, etc.

	\begin{table}[htb]
		\begin{center}
			\begin{tabular}{ccccc}
				\textbf{Precedencia} & \textbf{Operador} & \textbf{Acción}
				& \textbf{Tipos de datos} & \textbf{Ejemplos}\\
				\hline
				2 & * & Multiplicación & enteros, reales, valores complejos & y = 2 * x\\
				2 & / & División & enteros, reales, valores complejos & y = x / 2\\
				2 & \ \% & Módulo & enteros & y = 3 \% 2\\
				2 & \& & AND a nivel de bit & enteros & y = 1 \& 1\\
				2 & \&\textasciicircum & Bit clear & enteros & y \& 3\\
				2 & $<<$ & Desplazamiento a la izquierda & entero $<<$ entero
				  sin signo & y = 2 $<<$ 1\\
				2 & $>>$ & Desplazamiento a la derecha & entero $>>$ entero sin
				  signo & y = 2 $>>$ 1\\
				1 & + & Suma & enteros, reales, valores complejos, strings & y = 3 + y\\
				1 & - & Resta & enteros, reales, valores complejos & y = 2 - x\\
				1 & \textbar & OR a nivel de bit & enteros & y = 1 \textbar\ 0\\
				1 & \textasciicircum & XOR a nivel de bit & enteros & y = 1 \textasciicircum 1\\
				\hline
			\end{tabular}
		\end{center}
		\caption{Tabla de operadores aritméticos}
		\label{tabopar}
	\end{table}

	La prioridad de los operadores determina cómo serán evaluados en una
	expresión sin paréntesis que delimite esa prioridad. Así pues, si hay dos
	operandos de distinta prioridad, se evaluará primero la operación de mayor
	prioridad. Si los operandos existentes son de la misma prioridad, la
	evaluación se realizará de izquierda a derecha.\\

	La columna \emph{Tipos de datos} de la tabla \ref{tabopar} indica sobre qué
	tipos de datos se pueden utilizar estos operadores.\\

	Los operadores \emph{+}, \emph{-} y \emph{\textasciicircum} pueden ser
	utilizados de forma unaria, recibiendo un único operando. Veamos esto con un
	pequeño ejemplo:

\begin{minipage}{17.1cm}
\begin{lstlisting}[language=go,numbers=none,caption=Ejemplo de operandos aritméticos unarios,label=oparitunarios]
+x 		// Implica 0 + x, o lo que es lo mismo, indicar que x es positivo.
-x 		// Implica 0 - x, o lo que es lo mismo, indicar que x es negativo.
^x 		// Es el complemento de x. Transforma los bits de x de 1 a 0 y los que estén en 0 a 1.
\end{lstlisting}
\end{minipage}

	\subsection{Operadores relacionales}

	Además de los operadores aritméticos, necesitamos una serie de operadores
	que nos permitan comparar dos valores para saber cuál es mayor que otro, si
	son iguales, etc. En este caso, los operadores relacionales nos aportan esta
	funcionalidad.\\

	En el caso de los operadores aritméticos existe una prioridad que determina
	qué operación se evalúa antes, pero eso no es necesario para los operadores
	relacionales que siempre se evaluarán de izquierda a derecha.\\

	Los operadores relacionales existentes en Go son los que se observan en la
	tabla \ref{oprelacionales}.

	\begin{table}[!htb]
		\begin{center}
			\begin{tabular}{cc}
				\textbf{Operador} & \textbf{Acción}\\
				\hline
				$<$ & Menor que\\
				$>$ & Mayor que\\
				$<$= & Menor o igual que\\
				$>$= & Mayor o igual que\\
				$=$$=$ & Igual\\
				!= & Distinto\\
				\hline
			\end{tabular}
		\end{center}
		\caption{Tabla de operadores relacionales}
		\label{oprelacionales}
	\end{table}

	El resultado de una operación con operadores relacionales es siempre un
	\emph{Bool}, es decir, un valor \emph{true} o \emph{false}.

	\subsection{Operadores lógicos}

	Los operadores lógicos son aquellos que nos permiten realizar operaciones
	entre dos conjuntos de elementos, como se realiza en la lógica formal. De
	esta forma podremos evaluar una expresión en base a si se cumplen dos
	conjuntos de condiciones, si se cumple uno de ellos, o si no se cumple un
	conjunto dado.\\

	Los operadores lógicos en Go son:

	\begin{table}[htb]
		\begin{center}
			\begin{tabular}{ccc}
				\textbf{Prioridad} & \textbf{Operador} & \textbf{Acción}\\
				\hline
				3 & ! & Negación\\
				2 & \&\& & Conjunción (Y)\\
				1 & \textbar\textbar & Disyunción (O)\\
				\hline
			\end{tabular}
		\end{center}
		\caption{Tabla de operadores lógicos}
	\end{table}

	\subsection{Operadores a nivel de bit}

	Muchas veces es interesante poder manejar datos a nivel de bit\footnote{Un
	bit es la unidad mínima de representación de un ordenador y toma valores
	0 o 1.}	y realizar operaciones con ellos.

	Los operadores a nivel de bit disponibles en Go son:

	\begin{table}[htb]
		\begin{center}
			\begin{tabular}{ccc}
				\textbf{Prioridad} & \textbf{Operador} & \textbf{Acción}\\
				\hline
				2 & $<<$ & Desplazamiento a la izquierda\\
				2 & $>>$ & Desplazamiento a la derecha\\
				2 & \& & AND a nivel de bit\\
				2 & \&\textasciicircum & Bit clear (pone un bit a 0)\\
				1 & \textbar & OR a nivel de bit\\
				1 & \textasciicircum & XOR a nivel de bit\\
				\hline
			\end{tabular}
		\end{center}
		\caption{Tabla de operadores a nivel de bit}
	\end{table}

	\subsection{Asignaciones}

	Las asignaciones son fáciles y se realizan al igual que en la mayoría de
	lenguajes modernos. Para ello, se utiliza el operador `$=$'. Hay que
	recordar, que el operador de igualdad en Go, es `$==$'.\\

	Respecto a las asignaciones hay que tener en cuenta, que Go posee tres tipos
	de asignaciones o asignaciones con tres tipos de características:

	\begin{itemize}
		\item \textbf{Asignaciones simples}: A una variable se le asigna un
		valor constante o el valor de otra variable.
		\item \textbf{Asignaciones múltiples}: A un grupo de variables se les
		asignan	distintos valores simultáneamente.
		\item \textbf{Asignaciones múltiples de	funciones}: Las funciones, como
		veremos más adelante (sección \ref{funciones}), pueden devolver	valores
		múltiples, que pueden ser ``recogidos'' por varias variables.
	\end{itemize}

	Veamos un ejemplo de lo anterior:

\begin{minipage}{17.1cm}
\begin{lstlisting}[language=go,numbers=none,caption=Ejemplo de
asignaciones,label=asig]
a = 5
b = a

x, y, z = f(), g(), h()
a, b = b, a

nbytes, error := Write(buffer)

// Nótese que nbytes y error estarían siendo declaradas
// en ese instante (operador :=). Funciona igual para '='.
\end{lstlisting}
\end{minipage}

	\subsection{Operadores de comunicación}

	Como ya se ha mencionado en repetidas ocasiones, Go es un lenguaje muy
	orientado a la programación concurrente y paralela. Este tipo de paradigmas
	nos exige realizar operaciones de sincronización y comunicación entre
	tareas, para cumplir el objetivo final.\\

	Dada la simplicidad con la que se ha ideado el lenguaje existe un único
	tipo de datos, los \emph{canales} o \emph{channels} que estudiaremos con más
	profundidad en el la sección \ref{channels}.\\

	Este tipo de datos tiene un operador especial \emph{flechita} $<-$ que
	indica el sentido en el que se realiza la comunicación. Este operador recibe
	dos operandos y, al menos, uno de ellos debe ser de tipo \emph{chan}. El
	tipo de datos del otro operando dependerá de cómo se haya declarado el
	canal, ya que los canales manejan tipos de datos concretos.\\

	El operador $<-$ puede actuar de forma unaria, recibiendo un único
	parámetro, en cuyo caso el parámetro debe de ser un canal e implica la
	recepción del valor que almacene, descartándolo.\\

	Veamos un ejemplo de declaración de canales y uso de este operador, aunque se podrán
	ver más en la sección \ref{channels}.\\

\begin{minipage}{17.1cm}
\begin{lstlisting}[language=go,numbers=none,caption=Ejemplo de comunicación
entre canales, label=opflechita]
var c chan int 		// Declaramos un canal de números enteros
c <- 1 				// Envía un 1 al canal

v := <- c 			// Recibe el valor del canal (un 1 en este ejemplo), e inicializa v.
<- c 				// Recibe el valor del canal, descartándolo
\end{lstlisting}
\end{minipage}

	\subsection{Operadores de indirección}

	Go, como otros muchos lenguajes, implementa como otros muchos lenguajes un
	sistema de manejo eficiente de la memoria mediante los denominados
	\emph{Punteros}.\\

	Un puntero es, normalmente, una variable que apunta a una dirección de
	memoria que, normalmente, ocupa otra variable. De esta forma podemos manejar
	de manera mucho más eficiente el gasto de memoria de nuestro programa. En la
	sección \ref{newvsmake} veremos cómo se crean punteros a estructuras de
	datos complejos. Normalmente basta con referenciar un puntero a una variable
	con los operandos que vamos a ver en esta sección.\\

	El operando \emph{*} nos permite acceder a los datos apuntados por nuestra
	variable de tipo puntero, mientras que el operador \emph{\&} permite conocer
	cuál es la dirección de memoria de una variable o un valor determinado.
	Vamos a verlo más claro con un ejemplo:\\

\codigo{Ejemplo de operadores de
indirección}{label=opindireccion}{caracteristicas/ejemplopuntero.go}

	\subsection{Ejemplos}

	Aunque se puede encontrar mucha más referencia sobre los operadores en
	\cite{GoSpecOperators}, veamos en un cuadro resumen todos los operadores
	binarios ordenados de mayor a menor precedencia.

	\begin{table}[htb]
		\begin{center}
			\begin{tabular}{ccc}
				\textbf{Precedencia} & \textbf{Operadores} & \textbf{Comentarios}\\
				\hline
				5 & * \ / \ \% \ $<<$ \ $>>$ \ \& \&\textasciicircum & \&\textasciicircum\ significa "bit clear"\\
				4 & $+$ \ $-$ \ \textbar\ \ \textasciicircum & \textasciicircum\ significa "xor"\\
				3 & $==$ \ !$=$ \ $<$ \ $<=$ \ $>$ \ $>=$ & \\
				2 & \&\& & \\
				1 & \textbar\textbar & \\
			\end{tabular}
		\end{center}
		\caption{Tabla de operadores binarios}
	\end{table}

	En el siguiente ejemplo de código hay varios ejemplos de operaciones válidas.

\begin{minipage}{17.1cm}
\begin{lstlisting}[language=go,numbers=none, caption=Operaciones válidas, label=ejops]
+x
25 + 6*x[i]
y <= f()
^a >> b
f() || g()
x == y + 2 && <- chan_ptr > 0
x &* 7     // x con los 3 bits inferiores a 0.
fmt.Printf("\%5.2g\n", 2*math.Sin(PI/8))

"Hola" + "," + "adios"
"Hola, " + str
\end{lstlisting}
\end{minipage}

	\subsection{Diferencias con C}

	Los programadores de C que comiencen a programar en Go, se encontrarán
	varios cambios en lo referente a los operadores, y que son de bastante
	importancia.

	\begin{itemize}
		\item Go posee menos niveles de precedencia.
		\item \textasciicircum\ sustituye a $\sim$, que pasdda de ser un ``or
		exclusivo" binario, a ser unario.
		\item $++$ y $--$ no son operadores de expresiones.

			x$++$ es una instrucción, no una expresión;

			*p$++$ es (*p)$++$, no *(p$++$).
		\item \&\textasciicircum \ es un nuevo operador. Útil en expresiones
		constantes.
		\item $<<$, $>>$, etc. necesitan un contador de rotación sin signo.
	\end{itemize}

\section{Conversiones}

Como ya se ha mencionado anteriormente, las conversiones entre distintos tipos
deben realizarse de manera explícita, por lo que cualquier intento de conversión
implícita fallará en tiempo de compilación.\\

Así pues, y tal y como se mencionó al ver los tipos de datos numéricos,
convertir valores de un tipo a otro es una conversión explícita, que se realiza
como si fuera la llamada a una función. Veamos algún ejemplo:

\begin{minipage}{17.1cm}
\begin{lstlisting}[language=go,numbers=none]
uint8(int_var)		// truncar a 8 bits
int(float_var)		// truncar a a la parte entera
float64(int_var)	// cambio a coma flotante de 64 bits
\end{lstlisting}
\end{minipage}

También es posible realizar conversiones a \emph{string}, de manera que
podemos realizar las siguientes operaciones, entre otras:

\begin{minipage}{17.1cm}
\begin{lstlisting}[language=go,numbers=none]
string(0x1234)			// == "\u1234"
string(array_de_bytes)	// bytes -> bytes
string(array_de_ints)	// ints -> Unicode/UTF-8
\end{lstlisting}
\end{minipage}

\section{Valores constantes y números ideales}

Las constantes numéricas son ``números ideales": aquellos números que no tienen
tamaño ni signo, y por lo tanto no tienen modificadores \emph{l}, \emph{u},
o \emph{ul}.

\begin{minipage}{17.1cm}
\begin{lstlisting}[language=go,numbers=none]
077 			// Octal
0xFE8DACEFF 	// Hexadecimal
1 << 100
\end{lstlisting}
\end{minipage}

Existen tanto números ideales enteros como de coma flotante. La sintaxis del
literal utilizado determinará su tipo en tiempo de compilación.

\begin{minipage}{17.1cm}
\begin{lstlisting}[language=go,numbers=none]
1.234e5 	// float
1e2 		// float
630 		// int
\end{lstlisting}
\end{minipage}

Los números constantes en coma flotante y los enteros pueden ser combinados
según nos convenga. El tipo del resultado de las operaciones, así como las
propias operaciones, dependerá del tipo de los números utilizados. Así pues
tendremos:

\begin{minipage}{17.1cm}
\begin{lstlisting}[language=go,numbers=none]
2*3.14 		// float: 6.28
3./2 		// float: 1.5
3/28 		// int: 1

// Alta precisión:
const Ln2 = 0,693147180559945309417232121458\
			176568075500134360255254120680009

const Log2E = 1/Ln2 	// mayor precisión
\end{lstlisting}
\end{minipage}

\nota{Nótese la diferencia existente entre la imposibilidad de tener
conversiones entre tipos implícita, que impide el lenguaje, y la conversión
realizada en las operaciones, que sí que está definida. Esto es, a una variable
de tipo \emph{int}, no se le puede asignar el valor de una variable de tipo
\emph{float32} o \emph{float64}, sin una conversión de tipos previa, pero sí una
constante que represente un número real.}

La existencia de los números ideales tiene una serie de consecuencias, como la
vista en la nota anterior. El lenguaje define y permite el uso de constantes sin
una conversión explícita si el valor puede ser representado.

\begin{minipage}{17.1cm}
\begin{lstlisting}[language=go,numbers=none]
var million int = 1e6 	// Variable int, constante float
math.Sin(1)				// Pasamos un int, espera un float
\end{lstlisting}
\end{minipage}

Las constantes deben ser representables en el tipo al que queremos asignarlas.
Por ejemplo: \textasciicircum 0 es $-$1, que no es representable en el rango
0$-$255.

\begin{minipage}{17.1cm}
\begin{lstlisting}[language=go,numbers=none]
uint8(^0)		// Mal: -1 no es representable
^uint8(0)		// OK
uint8(350)		// Mal: 350 no es representable
uint8(35.0)		// OK: 35.0 = 35
uint8(3.5)		// Mal: 3.5 no es representable
\end{lstlisting}
\end{minipage}

\section{Declaraciones}

Como ya hemos comentado anteriormente, todas las declaraciones en Go van
precedidas de una palabra clave (var, const, type, func), seguidas del nombre
identificador que se quiera dar, más el tipo de datos en caso de ser una
variable y, opcionalmente, de una expresión de inicialización.\\

Los identificadores de variables, funciones, constantes, tipos, etc. se definen
por una cadena de caracteres alfanuméricos (letras y números) incluyendo el
carácter `\_'. La única restricción respecto a los identificadores es que no
pueden comenzar con un número.\\

Veamos algunos ejemplos de declaraciones en Go:

\begin{minipage}{17.1cm}
\begin{lstlisting}[language=go,numbers=none, caption=Declaraciones, label=declaraciones]
var i int = 0
const PI = 22./7.
type Punto struct {
	x, y int
}
func suma (a, b int) int {
	return a + b
}
\end{lstlisting}
\end{minipage}

Mucha gente se realiza la siguiente pregunta al ver por primera vez una
declaración en Go: ¿Por qué está invertido el tipo y la variable? Es
una cuestión de diseño del lenguaje, que otorga muchas facilidades para el
compilador, pero que además permite tener una mayor legibilidad o definir varios
punteros en una misma expresión sin olvidarse de añadir el carácter *. Por ejemplo:

\begin{minipage}{17.1cm}
\begin{lstlisting}[language=go,numbers=none]
var p, q *int
\end{lstlisting}
\end{minipage}

Además, las funciones son más cómodas de leer y son más consistentes con otras
declaraciones. Más adelante iremos viendo más razones para la inversión del tipo
respecto a otros lenguajes en la declaración.

	\subsection{Variables}

	Las variables en Go se declaran precedidas de la palabra reservada
	\emph{var}.\\

	Pueden tener un tipoi asociado, una expresión que inicialice la variable, o ambas
	cosas. Las expresiones que inicialicen las variables deben encajar con las
	variables y su tipo, ya que recordemos que Go es un lenguaje de tipado
	fuerte.\\

	Además, hay que tener en cuenta que podemos declarar varias variables de un
	mismo o distintos tipos en la misma línea, separando los nombres de las
	variables por comas.\\

	Veamos algunos ejemplos:

\begin{minipage}{17.1cm}
\begin{lstlisting}[language=go,numbers=none, caption=Declaraciones de variables,
label=declvar]
var i int
var j = 365.245
var k int = 0
var l, k uint64 = 1, 2
var entero, real, cadena = 1, 2.0, "Hola"
\end{lstlisting}
\end{minipage}

	Dado que puede ser bastante tedioso estar escribiendo todo el rato la
	palabra reservada \emph{var}, Go posee una característica muy interesante,
	que permite definir varias variables agrupadas por paréntesis
	y separándolas por puntos y coma. He aquí uno de los pocos usos obligatorios
	de dicho delimitador.

\begin{minipage}{17.1cm}
\begin{lstlisting}[language=go,numbers=none, caption=Declaración de variables agrupadas,
label=declvaragrup]
var (
	i int;
	j = 365.245;
	k int = 0;
	l, k uint64 = 1, 2;
	entero, real, cadena = 1, 2.0, "Hola"
)
\end{lstlisting}
\end{minipage}

	\nota{La última declaración de variables no tiene un ';' dado que
	no es obligatorio (ni aconsejable) ponerlo en dicha línea, ya que va antes
	de un carácter ')', tal y como se comentó en el apartado \ref{sintaxis}.}

	Este mismo esquema, no sólo sirve para definir variables, sino que también
	puede ser usado para definir constantes (\emph{const}) y tipos
	(\emph{type}).\\

	En las funciones, las declaraciones de la forma:

\begin{minipage}{17.1cm}
\begin{lstlisting}[language=go,numbers=none]
var v = valor
\end{lstlisting}
\end{minipage}

	pueden ser acortadas a:

\begin{minipage}{17.1cm}
\begin{lstlisting}[language=go,numbers=none]
v := valor
\end{lstlisting}
\end{minipage}

	Y teniendo las mismas características y tipado fuerte existente. He aquí,
	otro de los motivos para la inversión del nombre de la variable y el tipo de
	la misma.\\

	El tipo es aquél que tenga el valor, que para números ideales sería
	\emph{int}, \emph{float32} o \emph{float64} dependiendo del número utilizado. Este tipo de
	declaraciones de variables es muy usado y está disponible en muchos sitios,
	como en las cláusulas de los bucles \emph{for}.

	\subsection{Constantes}

	Las declaraciones de constantes están precedidas por la palabra reservada
	\emph{const}.\\

	Todas las constantes son inicializadas en tiempo de compilación, lo que
	significa que todas las declaraciones de constantes deben ser inicializadas
	con una expresión constante. Además, de forma opcional, pueden tener un tipo
	especificado de forma explícita.

\begin{minipage}{17.1cm}
\begin{lstlisting}[language=go,numbers=none,caption=Declaración de constantes,
label=declconst]
const Pi = 22./7.
const AccuratePi float64 = 355./113
const pepe, dos, comida = "Pepe", 2, "carne"

const (
	Lunes, Martes, Miercoles = 1, 2, 3;
	Jueves, Viernes, Sabado = 4, 5, 6
)
\end{lstlisting}
\end{minipage}

		\subsubsection{Iota}

		Las declaraciones de constantes pueden usar un contador especial,
		denominado \emph{iota}, que comienza en 0 en cada bloque declarado con
		\emph{const} y que se incrementa cada vez que encuenta un caracter
		punto y coma.

\begin{minipage}{17.1cm}
\begin{lstlisting}[language=go,numbers=none,caption=Declaración de constantes
con iota, label=decliota]
const (
	Lunes = iota;	// 0
	Martes = iota;	// 1
	iercoles = iota // 2
)
\end{lstlisting}
\end{minipage}

		Existe una versión mucho más rápida para el uso del contador especial
		\emph{iota}, que se basa en que la declaración de una variable pueda
		seguir un mismo patrón de tipo y expresión. Sería algo así
		como lo que se puede observar en el siguiente ejemplo:

\begin{minipage}{17.1cm}
\begin{lstlisting}[language=go,numbers=none,caption=Declaración de constantes
con iota - forma corta, label=decliota2]
const (
	loc0, bit0 uint32 = iota, 1<<iota;	// 0, 1
	loc1, bit1;							// 1, 2
	loc2, bit2                          // 2, 4
)
\end{lstlisting}
\end{minipage}

	\subsection{Tipos}

	El programador es libre de definir nuevos tipos a lo largo de su programa.
	Para declarar un nuevo tipo, se debe poner la palabra reservada \emph{type}.\\

	Más adelante, dedicaremos todo el capítulo \ref{datos_compuestos} a hablar sobre los distintos
	tipos de datos compuestos que el usuario puede crear, por lo que un simple
	ejemplo nos dará una idea de cómo se realiza la declaración de tipos.

\begin{minipage}{17.1cm}
\begin{lstlisting}[language=go,numbers=none,caption=Declaración de nuevos
tipos,label=declnuevostipos]
type Punto struct {
	x, y, z			float64
	identificador 	string
}
\end{lstlisting}
\end{minipage}

	\subsection{Operador new vs. función make\label{newvsmake}}

	En Go existe un operador \emph{new()} que básicamente reserva memoria para
	una variable. La sintaxis, es igual que la llamada a una función, con el
	tipo deseado como argumento, de igual forma que en lenguajes como C++.\\

	El operador \emph{new()} devuelve un puntero al objeto que se ha creado en
	memoria. Hay que notar, que a diferencia de otros lenguajes, Go no posee un
	\emph{delete} o un \textit{free} asociado, ya que para eso está el Garbage
	Collector, del que se habló en la sección \ref{Garbage Collector}.

\begin{minipage}{17.1cm}
\begin{lstlisting}[language=go,numbers=none,caption=Ejemplo del operador
new(),label=opnew]
var p *Punto = new(Punto)
v := new(int)				// v tiene como tipo *int
\end{lstlisting}
\end{minipage}

	Existe a su vez, una función llamada \emph{make} que se usará generalmente
	en tres tipos de datos concretos: maps, slices y channels. Normalmente en
	estos tipos queremos que los cambios hechos en una variable no afecten al
	resto, y por lo tanto nos aseguraremos que al crear este tipo de datos, se
	creen nuevas variables y no punteros a las mismas.\\

	Así pues para crear variables de tipos de datos complejos usaremos
	\emph{make} para los tipos anteriormente mencionados, que nos devolverá
	una estructura creada, o \emph{new} si usamos otros tipos de datos, ya que
	nos devolverá un puntero a dicha estructura.

